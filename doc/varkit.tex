\documentclass[12pt]{article}
\usepackage{hyperref}
\title{The Variation Toolkit}
\author{Pierre Lindenbaum PhD.}
\date{October 11, 2011}

\begin{document}
\maketitle
\section{Introduction}
The Variation Toolkit is a set of C/C++ programs to handle Variant Call Format (VCF).
The programs have been preliminary designed for knime4bio ( \url{http://code.google.com/p/knime4bio/})
but here, the bioinformaticians are the preliminary audience for this toolkit.
\section{Building}
\subsection{Dependencies}
\begin{description}
\item[wget]: wget is used to download the sources for samtools and tabix.
Your web proxy should allow an access to sourceforge.net .
\item[mysql-dev]: the files and libraries for mysql.
\item[libxml2]: the C library for xml \url{http://xmlsoft.org/}.
\item[libxslt]: the C library for xslt \url{http://xmlsoft.org/XSLT/}.
\item[libcurl]: the C library for downloading some URLs.
\end{description}

\section{Tools}

\subsection{ncbiefetch}
Fetch a record from NCBI database.

Currently supported databases: pubmed.

The following example generates a sequences of 6 pubmed ID and we call ncbiefetch to download the records.
\begin{quote}
\begin{verbatim}
$  (echo "#GI"; seq 1000 2 1010)   |\
      ncbiefetch -c 1 |\
      cut -c 1-100

#GI	pubmed.year	pubmed.title	pubmed.journal	pubmed.abstract
1000	1976	The amino acid sequence of Neurospora NADP-specific glutamate dehydrogenase. The tryptic p
1002	1976	The amino acid sequence of Neurospora NADP-specific glutamate dehydrogenase. Peptic and ch
1004	1976	Properties of 5-aminolaevulinate synthetase and its relationship to microsomal mixed-funct
1006	1976	The attachment of glutamine synthetase to brain membranes.	Biochemical medicine	...
1008	1976	Nature and possible origin of human serum ribonuclease.	Biochemical and biophysical resear
1010	1976	Formation of non-amidine products in the chemical modification of horse liver alcohol dehy
\end{verbatim}
\end{quote}

\subsection{ncbiesearch}
Search NCBI/Entrez:

The following example creates a sequence of 3 names, we search the NCBi for each name and the word "Rotavirus" in the title, limit to 2 record, we fetch each record (the PMID is in the 2nd column) and we cut the result down to 80 characters.
 
\begin{quote}
\begin{verbatim}
$ echo -e "#subject\nPiron\nLindenbaum\nPoncet" |\
   ncbiesearch -q '$1 "Rotavirus"[TITL]' -L 2  |\
   ncbiefetch -c 2 |\
   cut -c 1-80
#subject	pubmed.id	pubmed.year	pubmed.title	pubmed.journal	pubmed.abstract
Piron	10888646	2000	Efficient translation of rotavirus mRNA requires simultaneou
Piron	10364288	1999	Identification of the RNA-binding, dimerization, and eIF4GI-
Lindenbaum	15047801	2004	RoXaN, a novel cellular protein containing TPR, LD, and
Lindenbaum	8985320	1997	In vivo and in vitro phosphorylation of rotavirus NSP5 c
Poncet	21864538	2011	Structural Organisation of the Rotavirus Nonstructural Prot
Poncet	20935207	2010	Rapid generation of rotavirus-specific human monoclonal ant
\end{verbatim}
\end{quote}


\subsection{vcfttview}
\begin{quote}
\begin{verbatim}
$ echo -e "ref\t3\nref2\t2" |\
vcfttview -x 3 -B toy.bam -R toy.fa

>ref:3

1         11              21        31         41        51        61           
AGCATGTTAGATAA****GATA**GCTGTGCTAGTAGGCAG*TCAGCGCCATNNNNNNNNNNNNNNNNNNNNNNNNNNNN
      ........    ....  ......K.K......K. ..........                            
      ........AGAG....***...      ,,,,,    ,,,,,,,,,                            
        ......GG**....AA                                                        
        ..C...**** ...**...>>>>>>>>>>>>>>T.....                                 



>ref2:2

1         11            21        31        41        51        61              
aggttttataaaac****aattaagtctacagagcaactacgcgNNNNNNNNNNNNNNNNNNNNNNNNNNNNNNNNNNNN
.............Y    ..W...................                                        
..............****..A...                                                        
 .............****..A...T.                                                      
     .........AAAT.............                                                 
         C...T****....................                                          
           ..T****.....................                                         
             T****......................                                        
                                                               
\end{verbatim}
\end{quote}

\end{document}
